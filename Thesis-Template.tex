% Authors: Simon Geoffroy-Gagnon and Farhad Shokraneh
% Based on the Thesis by Rubana Bahar Priti
% Edit: 2020.04.16

\documentclass[12pt, TexShade, letterpaper]{report}
\usepackage[utf8]{inputenc}
\usepackage{palatino}
\usepackage{amsmath}
\usepackage{amssymb}
\usepackage{graphicx}
\usepackage[labelfont=bf]{caption}
\usepackage{subcaption}
\usepackage{setspace}
\captionsetup[table]{font = {stretch=1.35}}
\captionsetup[figure]{font = {stretch=1.35}}
\usepackage[margin=1in, headsep=1cm, bottom=5cm]{geometry}
\usepackage[hidelinks]{hyperref}
\newcommand{\MYhref}[3][blue]{\href{#2}{\color{#1}{#3}}}%
\usepackage{tabu}
\usepackage{cite}
\usepackage[table]{xcolor}
\usepackage{nomencl}
\usepackage[nonumberlist,nogroupskip,xindy]{glossaries}
\usepackage{floatrow}
\usepackage{wrapfig}
\renewcommand{\baselinestretch}{2}
\usepackage{fancyhdr}
\usepackage{lmodern}
\usepackage{listings}
\usepackage{diagbox} % for diagonal line command
\renewcommand{\chaptermark}[1]{\markboth{#1}{}} % Ensure List of Figs, ToC, and glossary are named in the header

\setglossarystyle{long}
\renewcommand{\glsnamefont}[1]{\textbf{#1}}

\newglossarystyle{mystyle}{%
	\renewenvironment{theglossary}%
	{\begin{longtabu} to \linewidth {p{0.15\linewidth}p{0.85\linewidth}}}%
		{\end{longtabu}}%
	\renewcommand*{\glossaryheader}{}%
	% indicate what to do at the start of each logical group
	\renewcommand*{\glsgroupheading}{}
	\renewcommand*{\glsgroupskip}{}%
	\renewcommand*{\glossaryentryfield}[5]{%
		\glstarget{##1}{##2}% Name
		& ##3% Description
		\\% end of row
	}
}

% Overwrite the plain page style with a red line and page numbering
\fancypagestyle{plain}{%
	\fancyhf{} % clear all header and footer fields
	\fancyhead[R]{\textbf{\thepage}} % except the center
}

% Create the fancy page style header
\pagestyle{fancy}
\fancyhf{}
\lhead{\textbf{\nouppercase{\leftmark}}}
\chead{}
\rhead{\textbf{\thepage}}


\usepackage{xpatch}
\xpretocmd\headrule{\color{red}}{}{\PatchFailed}

% Get rid of all dashed words
\tolerance=1
\emergencystretch=\maxdimen
\hyphenpenalty=10000
\hbadness=10000

% Create glossary
\newacronym{lm}{LM}{Levenberg-Marquardt}
\newacronym{mcmc}{MCMC}{Markov Chain Monte Carlo}
\newacronym{dls}{DLS}{Damped Least-Squares}
\newacronym{ares}{ARES}{The Accelerated Reionization Era Simulations}
\newacronym{edges}{EDGES}{Experiment to Detect the Global EoR Signature}
\newacronym{eor}{EoR}{Epoch of Reionization}
\newacronym{igm}{IGM}{Intergalactic Medium}
\newacronym{sfr}{SFR}{Star Formation Rate}
\newacronym{saras}{SARAS}{Small Array for Research in Astrophysics of the South}
\newacronym{cmb}{CMB}{Cosmic Microwave Background}
\newacronym{prizm}{PRIZM}{Probing Radio Intensity at high-Z from Marion}
\makeindex
\makeglossaries

% Set page numbering to roman
\setcounter{page}{2}\renewcommand{\thepage}{\roman{page}}

\author{\textcopyright Author, August, 2020}
\date{}

\begin{document}

\begin{titlepage}
		\begin{center}
			\vspace*{0.5cm}

			\LARGE
			\textbf{Parameter Estimation of Global 21cm Signal}
			%Probing the Effect of Non-Standard Physics on Future 21cm Observations
			\vspace{1cm}
			
			\textit{Aryana Haghjoo}
			
			\vspace{1cm}
			
			\includegraphics[width=0.6\textwidth]{McGill_logo.png}
			
			\vspace{1cm}
			
			\Large
			Department of Physics
			
			\vspace{-5mm}
			McGill University
			
			\vspace{-5mm}
			Trottier Space Institute
			
			\vspace{-5mm}
			Montr\'eal, Qu\'ebec, Canada
			
			\vspace{5mm}
			August 2023
			\small
			\vspace{0.5cm}
			{\color{red} \hrule height 0.75mm}
			
			\vspace{0.2cm}
			
			A thesis submitted to McGill University in partial fulfillment of the requirements of the degree of
			\emph{Master's of Science in Physics}
		
			\copyright\hspace{0.5mm}2023 Author
			
		\end{center}
	\end{titlepage}
\setlength{\voffset}{2cm}
\renewcommand{\chaptermark}[1]{%
	\markboth{\thechapter.\ #1}{}}
\chapter*{Abstract}\markboth{Abstract}{}
	\label{chap:engAbstract}
%	\addcontentsline{toc}{section}{\nameref{chap:engAbstract}}
The global 21cm signal has emerged as a crucial observable in cosmology and astrophysics, providing valuable insights in the of study of the period between the end of the cosmic dark ages and the formation of the first stars and galaxies.\par
The 21cm signal is sensitive to the density and temperature of neutral hydrogen in the early universe and the presence of the first stars and galaxies. Therefore, any deviation from the predictions of the standard cosmological model of this signal could indicate the presence of new physics beyond the standard model. In this study, we explore the potential of this signal to reveal non-standard physics by means of providing a new path to test fundamental physical theories. \par
The literature review  provides a comprehensive overview of the global 21cm signal by exploring the physical principles, simulations, imprints of non-standard effects, parameter estimation, and observation attempts associated with the global 21cm signal. The physical principles encompass the mechanisms involved in forming the global 21cm signal and its evolution through cosmic history. Simulations play a pivotal role in generating models of the global 21cm signal, aiding in understanding the influence of different astrophysical scenarios on the ultimate behaviour of this signal. Furthermore, the effects of non-standard physics on the global 21cm signal are examined, encompassing scenarios such as cosmic strings, exotic particle interactions, or additional dark matter components. Additionally, parameter estimation techniques are discussed, highlighting the methodologies employed to extract valuable astrophysical information from observed 21cm data. The review also delves into the ongoing efforts and challenges in observing the global 21cm signal and complexities of foreground removal.\par
Ultimately, this thesis focuses on a specific parameter estimation method. We adopt the Markov Chain Monte Carlo (MCMC) method combined with the Levenberg Marquardt (LM) algorithm to estimate the best-fit physical parameters (e.g., clumping factor, star formation efficiency) of the theoretical 21cm curves. We use the Accelerated Reionization Era Simulations (ARES) to generate models for the global 21-cm signal. La méthode susmentionnée nous donne un contrôle total sur les paramètres de l'algorithme, ce qui permet de sélectionner n'importe quelle combinaison de paramètres disponible dans le cadre d'ARES.\par
The knowledge of these best-fit parameters will help us to constrain future proposed models and set theoretical limits for the precision of upcoming experiments to observe non-standard effects.\par
\chapter*{Résumé}\markboth{Résumé}{}
	\label{chap:frAbstract}
%	\addcontentsline{toc}{section}{\nameref{chap:frAbstract}}
Le signal global à 21 cm est devenu une observable cruciale en cosmologie et en astrophysique, fournissant des informations précieuses pour l'étude de la période entre la fin des âges sombres cosmiques et la formation des premières étoiles et galaxies.\par

Le signal à 21 cm est sensible à la densité et à la température de l'hydrogène neutre dans l'univers primitif et à la présence des premières étoiles et galaxies. Par conséquent, tout écart par rapport aux prédictions du modèle cosmologique standard de ce signal pourrait indiquer la présence d'une nouvelle physique au-delà du modèle standard. Dans cette étude, nous explorons le potentiel de ce signal à révéler une physique non standard en fournissant une nouvelle voie pour tester les théories physiques fondamentales. \par

La revue de la littérature fournit une vue d'ensemble du signal global de 21 cm en explorant les principes physiques, les simulations, les empreintes d'effets non standard, l'estimation des paramètres et les tentatives d'observation associées au signal global de 21 cm. Les principes physiques englobent les mécanismes impliqués dans la formation du signal global à 21 cm et son évolution au cours de l'histoire cosmique. Les simulations jouent un rôle essentiel dans la génération de modèles du signal global de 21 cm, aidant à comprendre l'influence de différents scénarios astrophysiques sur le comportement final de ce signal. En outre, les effets de la physique non standard sur le signal global de 21 cm sont examinés, englobant des scénarios tels que les cordes cosmiques, les interactions de particules exotiques ou des composants supplémentaires de matière noire. En outre, les techniques d'estimation des paramètres sont discutées, mettant en évidence les méthodologies employées pour extraire des informations astrophysiques précieuses des données observées à 21 cm. La revue se penche également sur les efforts et les défis en cours dans l'observation du signal global de 21cm et sur les complexités de l'élimination de l'avant-plan.\par

En fin de compte, cette thèse se concentre sur une méthode spécifique d'estimation des paramètres. Nous adoptons la méthode de Monte Carlo par chaîne de Markov (MCMC) combinée à l'algorithme de Levenberg Marquardt (LM) pour estimer les paramètres physiques les mieux ajustés (par exemple, le facteur d'agglomération, l'efficacité de la formation d'étoiles) des courbes théoriques à 21 cm. Nous utilisons les simulations de l'ère de réionisation accélérée (ARES) pour générer des modèles pour le signal global de 21 cm. Notre méthode est flexible quant au choix des paramètres d'ARES.\par

La connaissance de ces paramètres les mieux adaptés nous aidera à contraindre les modèles proposés à l'avenir et à fixer des limites théoriques pour la précision des expériences à venir visant à observer des effets non standard.\par

\chapter*{Acknowledgements}\markboth{Acknowledgements}{}
	\label{chap:acknowledgments}
%	\addcontentsline{toc}{section}{\nameref{chap:acknowledgments}}


Reminder: name of my friends, my roommates, something in Persian if allowed, computational resources Niagara \par 
I would like to express my gratitude to my supervisors, Jonathan Sievers and Oscar Hernandez, for their invaluable support throughout this project. I am also deeply grateful to my parents for their unwavering support, even from thousands of kilometers away. Additionally, I want to thank my new friends in Montreal for the wonderful experiences we shared.\par
Furthermore, I would like to acknowledge the significant role that my therapist at McGill Wellness Hub, Romeo Bidar, played in my graduate journey. His constant presence was a true blessing.\par
 To all the lifelong learners and doers: you are changing the world every day.
 % Start of ToC, LoT, gls
	\tableofcontents\thispagestyle{plain}

	\listoffigures\thispagestyle{plain}
%	\addcontentsline{toc}{section}{\listfigurename}
	\listoftables
%	\addcontentsline{toc}{section}{\listtablename}
	\glsaddall
	\setlength\LTleft{0pt}
	\setlength\LTright{0pt}
	\setlength\glsdescwidth{0.8\hsize}
	\printglossary[title={List of Acronyms}]
	\markright{List of Acronyms} 

 	\clearpage
	\pagenumbering{arabic} % restart page numbers at one, now in arabic style
	
	\glsresetall
	% start of mainmatter
%###########################################################
\chapter{Introduction}
\label{chap:intro}
Reminder: The preface can be expanded, I should come back to writing this chapter after I am done with chapter 2 and 3. I guess it would help to have some addressing of those information here. \par
21cm cosmology in a relatively new opened window in the study of universe during its early states. It has been proven to be one of the only available probes of large-scale structure in the most distant reaches of the universe\cite{primordial_universe}.\par
In this chapter we will talk bout the importance of this signal and its applications. Then, we will describe the motivations of this research to clarify the effects of non-standard physics on this signal. It is worth mentioning that we will only focus on the global 21cm signal.\par
Finally, in the last section of this chapter, we will briefly go over all the materials provided in this thesis.\par
%-----------------------------------------------------------
\section{Background and motivation}
Not very long after the big bang ($200 \lesssim z \lesssim 1100$), before the formation of stars and galaxies, the universe was composed of huge clouds of gas ($75\%$ hydrogen, the rest helium and small amounts of heavy elements\cite{21century}), and the remaining photons from the big bang called cosmic microwave background (CMB) \cite{map_universe}. \par
Since neutral hydrogen is the most frequent component of the intergalactic medium (IGM), it provides us a convenient tracer for the behaviour of baryonic matter in the early universe. Neutral hydrogen has a hyperfine splitting on its 1S ground state caused by the interaction between the magnetic moment of the proton and electron ($\Delta E = 5.9 \times 10 ^{-6}eV$\cite{21century}). This spin-flip transition results in absorption or emission of a radio signal with the frequency of $1420.4057MHz$\cite{low_frequency} corresponding to a wavelength of $21.1cm$\cite{21century}\footnote{The existence of this spectral line was theoretically predicted by van de Hulst in 1942 and the first ever detection was reported by Ewen and Purcell in 1951\cite{21century}.}.\par
As the universe undergoes expansion, the redshifting of CMB photons leads to a specific phase where their wavelengths approach those closely resembling the intrinsic bands of hydrogen. 
During this period, a significant phenomenon occurs as hydrogen molecules interact with CMB photons. This interaction takes place approximately between the dark ages and the epoch of reionization (EoR). As a consequence of this interaction, the brightness temperature of the 21cm line experiences a notable alteration\footnote{This interaction will be discussed in detail in Chapter \ref{chap:global21cm} as the coupling of the spin temperature ($T_s$) with the kinetic temperature of the gas ($T_k$).}. By employing modern radio interferometers, equipped with precise observation capabilities, this alteration can be effectively detected. It is worth noting that due to the redshift, the 21cm signal attains frequencies ranging from approximately 30 to 200 MHz, rendering it readily accessible to our radio interferometers\cite{low_frequency}.
These observations are capable of giving us valuable information about the distribution of neutral hydrogen and evolution of cosmic structure over time\cite{low_frequency}.\par
The global 21cm signal is the average over the brightness temperature of 21cm line across the entire sky. It is a measure of overall state of intergalactic medium (IGM) which might present as an excess absorption or emission in different redshift regions. This radiation is an observational evidence for certain characteristics of the IGM in the early universe (e.g. temperature, density and reionization state). These properties are determined by the complex interplay between the cosmic radiation field, the formation and evolution of the first stars and galaxies, and the feedback processes that these sources exert on their surroundings\cite{21century}.\par
Besides all the above-mentioned applications, the global 21cm signal is a \textbf{strong probe for non-standard physics} during the dark ages. It has the potential to shed a light on mysteries surrounding dark matter/dark energy, existence of cosmic strings, and even certain particle interaction (citation) \footnote{These effects will be discussed more in \ref{chap:global21cm,sub:non_standard}.}. This capacity of the global 21cm signal is the main motivation of this research.\par
%-----------------------------------------------------------
\section{Research questions and objectives}
Reminder: citations \par
The effects of non-standard physics in 21cm signal has gained lots of attraction in the recent literature (cite). Many reported effects has been investigated only using analytical methods, regardless of the observational data (citation).\par
We aim to fill this research gap by probing these effects using semi-analytical simulations of the global 21cm signal. With the recently released data at one's disposal (cite edges), we use certain fitting algorithms to compare them to theoretical simulated models (cite ares). We first estimate the physical parameters of theses curves by only taking the standard physical mechanisms into account.\par
In future studies, we will include a realistic foreground model. and upgrade our simulator to include arbitrary non-standard effects. We will again fit the observational data to the upgraded version of theoretical model to check if any of the best-fit physical parameters show patterns of change.\par
%-----------------------------------------------------------
\section{Overview of the thesis}
Reminder: citations\par
This thesis consists of two major components: the literature review and the research.\par
In the literature review chapters (chapters \ref{chap:global21cm} and \ref{chap:observations}), we first talk about the global 21cm signal and the its physical principals in chapter \ref{chap:global21cm}. In section \ref{chap:global21cm,sub:non_standard}, we explore the effects of non-standard physics on global 21cm signal mentioned in the literature. Finally, in section \ref{chap:global21cm,sub:parameter_estimation}, we talk about the importance of estimating the parameters of this signal and how these estimations affects future proposed observations. Furthermore, we briefly talk about different methods used to serve this purpose and advantages of each one.\par
In chapter \ref{chap:observations} we will briefly introduce all of the past, present and future observational project who aim to detect the global 21cm signal. From all these experiments, we will specifically talk about two of them in detail.\par
In the research half of this thesis, we will talk about the parameter estimation of global 21cm signal. The specific method used in this this study is described in chapter \ref{chap:method} thoroughly. Subsequently, we apply this method to fit the chosen observational data. The results are demonstrated in chapter \ref{chap:results}. Finally, in chapter \ref{chap:discussion}, we interpret the outcome and talk the implications of these results for cosmology and astrophysics. Also, the limitations and the future prospective of the project is mentioned in this chapter.\par
%###########################################################
\chapter{The Global 21cm Signal}
\label{chap:global21cm}
As previously mentioned, the global 21cm signal is a new window to extract data and build a better understanding of the early universe. To achieve this goal, we need to carefully construct the physical theory, try to simulate the signal, build radio telescopes to observe it and finally, analyse the observational data and compare them to our theoretical models.\par
In this chapter, we will present the latest findings about the physical background, simulations and effects of non-standard physics on global 21cm signal. In the last section of this chapter, we will talk about the importance of parameter estimation and its methods.\par
In chapter \ref{chap:observations}, we will talk about the attempts of observing this signal and foreground removal challenges.\par
%-----------------------------------------------------------
\section{Theoretical basis and physical principles of the 21cm signal}
\subsection{evolution of global 21cm signal through cosmic history}
Reminder: read the more precise papers and add the exact redshift ranges \par
400000 years after the Big Bang, the universe becomes cool enough for subatomic particles to combine and form neutral hydrogen. As a result, baryonic matter experiences thermal decoupling from the CMB, making it possible for the non-relativistic gas to cool adiabatically with the expansion of universe\cite{21century}. The temperature of the gas and CMB drops as $T\propto a^{-2}$ and $T\propto a^{-1}$ respectively\footnote{a is the expansion scale factor}. This rate discrepancy leaves the gas cooler than radiation, while effectively stopping the heat exchange between these two components. As a result, the gas spin temperate decouples from the CMB temperature. Since the gas density is sufficiently high, the effect of collisional coupling becomes significant and force the spin temperature to couple to the gas kinetic temperate ($T_S \rightarrow T_K$). Consequently, the 21cm brightness temperature becomes negative (compared to the CMB background) in the average spectrum corresponding to this redshift\cite{map_universe}.\par
As the density keeps dropping, the influence of collisional couplings becomes less important, causing the spin temperature to again couple with the CMB temperature ($T_S \rightarrow T_\gamma$). Thus, the 21cm brightness signal goes back to zero again\cite{map_universe}. The first blue regions in the upper panel of figure \ref{fig:global_signal_pritchard_loeb} and corresponding absorption through in the lower panel depict the above-mentioned process.\par
The second fluctuation in figure \ref{fig:global_signal_pritchard_loeb} is emerged though the formation of first stars and galaxies. As large halos begin to collapse to form first stars and galaxies, they generate $Ly\alpha$ photons through two mechanisms: 1) Continuum radiation redshifted to $Ly\alpha$ wavelength and 2) Ionization and deexcitation of IGM gas. This produced $Ly\alpha$ background, strongly couples the excitation of the 21cm line spin states and the gas spin temperature ($T_S \rightarrow T_K$) through the Wouthuysen-Field effect. Therefore, we will again observe the 21cm absorption through\cite{map_universe}\cite{21century}.\par
Eventually, the gas is reheated by ionizing x-ray photons from galaxies\cite{21century}, and the 21cm brightness temperature goes back to zero (second absorption through in the lower panel of figure \ref{fig:global_signal_pritchard_loeb}). The influence of this reheating mechanism is such strong that we even expect to observe 21cm emission in this redshift range (red region in the upper panel of figure \ref{fig:global_signal_pritchard_loeb}). From this point on in the thermal history of universe, we do not expect to see anymore 21cm signal from IGM since most of the gas will be ionized by emergent stars, galaxies and accreting blackholes. Therefore, the 21cm brightness temperature goes to zero (black region in the upper panel of figure \ref{fig:global_signal_pritchard_loeb}). However, in the regions with higher recombination rate inside galaxies and dark matter halos, the neutral hydrogen is shielded and preserved. Thus, its 21cm signature originating from inside the galaxies is still observable\cite{map_universe}.\par
\begin{figure}[h!]
\centering
\includegraphics[scale =0.8]{global_signal_pritchard_loeb.jpg}
\caption[Time evolution of the fluctuations in the 21cm brightness temperature]{Time evolution of the fluctuations in the 21cm brightness temperature compared to the CMB background (dashed line). The top panel is a color plot showing fluctuations arising from variation in density. The coloration clearly shows the two absorption phases of 21cm brightness temperature (coupling of $T_S$ to $T_K$, blue), emission phase (red), and the regions where it disappears completely (coupling of $T_S$ to $T_\gamma$, black). The lower panel represents the evolution of sky-average (global) 21cm signal from the dark ages to the reionization. The frequency range of the two absorption through and the emission regions exactly matches the  corresponding regions in the upper panel. Figure from Pritchard and Loeb, 2011 \cite{21century}}
\label{fig:global_signal_pritchard_loeb}
\end{figure}
%-----------------------------------------------------------
\section{Simulating the global 21cm signal}
\subsection{The Accelerated Reionization Era Simulations (ARES)}
\subsection{21cmFast}
%------------------------------------------------------------
\section{Effects of non-standard physics on the global 21cm signal}
\label{chap:global21cm,sub:non_standard}
\subsection{Cosmic Strings}
\subsection{Dark Matter}
\subsection{Other non-standard effects}
\section{Parameter estimation of global 21cm signal}
\label{chap:global21cm,sub:parameter_estimation}
talk about different methods used to serve this purpose and advantages of each one\par
While neural networks have been shown to be effective, one drawback is the lack of clear physical interpretation of the resulting parameter values. In contrast, sampling algorithms like Markov chain Monte Carlo (MCMC), which preserve the basis vectors of the parameter space, can overcome this limitation. The parameters used in these methods are directly linked to the underlying physical mechanisms of the model.
\subsection{Machine Learning}
\subsection{MCMC}
%##############################################################
\chapter{Observations of the Global 21cm Signal}
\label{chap:observations}
List of experiments (reminder: rephrasing and citing each of them,  things that I need for each experiment: dipole antenna, location, time frame of results, frequency range):\par
\begin{enumerate}
\item EDGES (Experiment to Detect the Global EoR Signature): This was the first experiment to report a detection of the global 21cm signal in 2018, by measuring the absorption of radio waves from distant sources by neutral hydrogen in the intergalactic medium.

\item SARAS (Small Array for Research in Astrophysics of the South): This is a radio telescope located in India that has been used to study the global 21cm signal since 2015.

\item SCI-HI (Small Radio Telescope for Cosmic Hydrogen Intensity Mapping): This is a low-frequency radio telescope located in the United States that was designed to detect the global 21cm signal during the epoch of reionization.

\item BIGHORNS (Broadband Instrument for Global HydrOgen ReioNisation Signal): This is a radio interferometer located in Australia that was designed to detect the global 21cm signal during the epoch of reionization.

\item HERA (Hydrogen Epoch of Reionization Array): This is a radio interferometer located in South Africa that is designed to detect the global 21cm signal during the epoch of reionization and cosmic dawn.

\item Tianlai: This is a radio interferometer located in China that is designed to detect the global 21cm signal during the epoch of reionization and cosmic dawn.

\item Murchison Widefield Array: This is a radio interferometer located in Australia that is used to study the early Universe, including the global 21cm signal.

\item LOFAR (Low-Frequency Array): This is a radio interferometer located in the Netherlands that is used to study a wide range of astrophysical phenomena, including the global 21cm signal.

\item SKA
\item REACH
\item PRIZM (Polarized Radiometer In Space) is another experiment designed to study the global 21cm signal. It is a small satellite mission developed by NASA's Goddard Space Flight Center in collaboration with the Korean Astronomy and Space Science Institute (KASI).

The PRIZM mission aims to detect the global 21cm signal from the cosmic dawn era using a radiometer operating at a frequency range of 30-50 MHz. The satellite was launched in December 2018 as a secondary payload on a SpaceX Falcon 9 rocket and is in a low-Earth orbit.

PRIZM's scientific goals include measuring the temperature and polarization of the cosmic microwave background radiation, studying the reionization epoch, and detecting the global 21cm signal. It is expected to be the first space-based mission to detect the global 21cm signal and could provide crucial insights into the early Universe.
These are just a few of the experiments designed to detect the global 21cm signal. Other experiments include the Canadian Hydrogen Intensity Mapping Experiment (CHIME), the Dark Ages Radio Explorer (DARE), and the Square Kilometer Array (SKA).
\end{enumerate}
%--------------------------------------------------------------
\section{Experiment to Detect the Global EoR Signature (EDGES)}
\subsection{error bars and foreground removal}
\subsection{theoretical model and associated parameters}
%---------------------------------------------------------------
\section{Small Array for Research in Astrophysics of the South (SARAS)}
%#################################################################
\chapter{Parameter Estimation Methods}
\label{chap:method}
We formerly discussed the parameter estimation and model fitting of global 21cm in \ref{chap:global21cm,sub:parameter_estimation}. Various computational algorithms, including machine learning and neural networks, have been used to accomplish this task. While neural networks have been shown to be effective, certain drawbacks put sampling algorithms like MCMC on an advantaged position for global 21cm applications.\par
In this chapter, we describe the main fitting algorithm used in our study, which combines Levenberg-Marquardt and MCMC. We explain the reason behind utilzing both of these algorithms for this particular fitting challenge and provide a detailed explanation of the procedure to combine them. We also discuss two different tests used to measure the quality of the proposed fit.\par
In chapter \ref{chap:results}, we present the results of applying this fitting method to the observed global 21cm curve\footnote{released by the EDGES group} and its corresponding theoretical model\footnote{generated using ARES}.
%------------------------------------------------------------------
\section{Levenberg-Marquardt (LM)}
\label{chap:method,sub:LM}
The Levenberg-Marquardt (LM) algorithm, also known as the damped least-squares (DLS) method, is a fitting algorithm used for non-linear least-squares problems. This iterative algorithm is based on another gradient decent method referred to as "Newton's method".\par
LM is perfectly capable of fitting models with Gaussian-shaped likelihood spaces. However, it's abilities are limited when it comes to more complicated surfaces.\par
We continue this chapter by deriving the basic analytical definition of this method. In order to do so, we start by defining the matrix form of chi-square. Subsequently, we attempt to calculate the second order derivative of chi-square with respect to the parameters of the model. We will show that this calculation leads to defining the covariance matrix, which will be later used as our basis to generate correlated noise (\ref{chap:method,sub:correlated noise}).\par
\subsection{Covariance Matrix}
\label{chap:method,sub:cov_mat}
The goal of gradient decent algorithms is to minimize the following sum called the "chi-square":
\begin{gather}
    \chi^2 \equiv \sum \frac {(x_i - \mu_i)^2}{\sigma^2_{i}}
    \label{eq: chi-sqaure simple}
\end{gather}
Where $x_i$ is the observed data, $\mu_i$ is the expected value with respect to the theoretical model, and $\sigma$ is the error associated with each data point. Is it often much easier to calculate the derivatives if we write the above expression in the language of linear algebra. $\mu$ can be defined as $\mu_i=<d_i>=A_i(m)$, where A is the model which is dependent on the set of parameters m (the dependency can be nonlinear). We can also define a diagonal noise matrix $N$, where $N_{i, i} = \sigma^2_{i}$. Substituting these two in equation \ref{eq: chi-sqaure simple} we get:
\begin{gather}
    \chi^2 = (d-A(m))^T N^{-1} (d-A(m))
    \label{eq:chi-square matrix}
\end{gather}
Where d is the array containing the data points.\par
We continue by calculating the first two derivatives of the above expression, leading to the construction of the gradient decent method.
\begin{gather}
        \frac{d \chi^2}{dm} = - (\frac{dA(m)}{dm})^T N^{-1} (d-A(m)) - (d-A(m))^T N^{-1} \frac{dA(m)}{dm} \label{eq:csq_first_deriv}
\end{gather}
Since we know that:
\begin{gather}
    (N^{-1})^T = N^{-1}\\
    [\frac{dA(m)}{dm} N^{-1} (d-A(m))]^T = (d-A(m))^T N^{-1} \frac{dA(m)}{dm}
\end{gather}
Substituting in \ref{eq:csq_first_deriv}, we get:
\begin{align}
    \frac{d \chi^2}{dm} = -2 (\frac{dA(m)}{dm})^T N^{-1} (d-A(m))\\
\end{align}
Thus, we can calculate the second derivative:
\begin{align}
    \frac{d^2 \chi^2}{dm^2} = -2 (\frac{d^2A(m)}{dm^2})^T N^{-1} (d - A(m)) -2 ( \frac{dA(m)}{dm}) ^T N^{-1} (-\frac{d\chi^2}{dm})
\end{align}
The first term can simply be neglected due to the following reasons:\par
\begin{enumerate}
    \item The $(d - A(m))$ component, which is the residual comparing the model to data, can take both negative and positive values; Thus on average, it will be close to zero.
    \item The best-fit model is expected to mimic the behavior of data closely. Therefore, we the residuals must have a small value: $(d - A(m)) \approx 0$
\end{enumerate}
Finally, using the above mentioned logic, we are left with:
\begin{align}
         \frac{d^2 \chi^2}{dm^2} = 2 (\frac{dA(m)}{dm})^T N^{-1} (\frac{dA(m)}{dm}) \label{eq:csq_second_deriv}
\end{align}
Which is the definition of the \textbf{Covariance Matrix}. The dimensions of this square matrix is the the same as that of $m$. The diagonal elements of this matrix are simply the inverse of variance on each parameter ($\sigma_{i, i}$), while the off-diagonal elements represent the inverse of covariance, measuring the dependency of a pair of parameters ($\sigma_{i, j}$). If calculated correctly, this matrix is always semi-positive definite\footnote{A positive definite matrix only posses positive eigenvalues. However, for a semi-positive definitive matrix, eigenvalues are non-negative.}.\par
%----------------------------------------------------------------------
\subsection{Derivation of the Levenberg-Marquardt Algorithm}
In LM method, on each iteration, the set of parameters m is replaced with $m+\delta m$. To find the $\delta m$, the function $\chi^2 (m +\delta m)$ is approximated by its linearization: 
\begin{gather}
    \chi^2 (m) = (d-A(m))^T N^{-1} (d-A(m))\\
    \chi^2 (m + \delta m) =  \chi^2 (m) + \frac{d \chi^2}{dm} \delta m \label{eq:csq_perturb_deriv}
\end{gather}
Similar to the procedure done in section \ref{chap:method,sub:LM}, we calculate the derivative of \ref{eq:csq_perturb_deriv}:
\begin{gather}
    \frac{d \chi^2 (m +\delta m)}{dm} = \frac{d}{dm} (\chi^2) + \frac{d}{dm} (\frac{d\chi^2}{dm} \delta m)
\end{gather}
We already have the expression for the first order derivative of chi-square in \ref{eq:csq_first_deriv}. Therefore:
\begin{gather}
   \frac{d \chi^2 (m +\delta m)}{dm} =  -2 (\frac{dA(m)}{dm})^T N^{-1} (d-A(m)) + \frac{d^2 \chi^2}{dm^2} \delta m + \frac{d\chi^2}{dm} \frac{d}{dm} (\delta m)
\end{gather}
Where the last term equals zero since $\delta m$ does not have any fundamental dependencies on $m$. Looking at the second term, it is simply inferred that we have already found the expression for second derivative of chi-square in \ref{eq:csq_second_deriv}. Thus, we are left with:
\begin{gather}
    \frac{d \chi^2 (m +\delta m)}{dm} =  -2 (\frac{dA(m)}{dm})^T N^{-1} (d-A(m)) + 2 (\frac{dA(m)}{dm})^T N^{-1} (\frac{dA(m)}{dm})
\end{gather}
We define  $d-A(m) \equiv r$, and $\frac{dA(m)}{dm} \equiv A'$, so the expression takes the following form:
\begin{gather}
    A'^{T} N^{-1}A' \delta m = A'^T N^{-1} r \\
    \delta m = (A'^{T} N^{-1}A')^{-1} A'^T N^{-1} r \label{eq:newton's_method}   
\end{gather}
Equation \ref{eq:newton's_method} represents the basis for \textbf{Newton's method}. However, as previously mentioned, this method suffers from convergence issues, especially on complicated likelihood surfaces. To overcome this obstacle, we add a new term to the left-hand side of \ref{eq:newton's_method}. This term includes a control parameter $\Lambda$ that is updated on each iteration depending on the quality of fit. \par
\begin{gather}
    (A'^{T} N^{-1}A' + \Lambda I)\delta m = A'^T N^{-1} r\\
    \delta m = (A'^{T} N^{-1}A' + \Lambda I)^{-1} A'^T N^{-1} r
\end{gather}
Now the basic idea is apparent: On each iteration, a set of parameters $m$ will be replaced by $m+\delta m$ and the chi-square is calculated based on the perturbed parameters. Subsequently, the new chi-square is compared to its value in the last step. If we encounter a higher value, the $\Lambda$ will be multiplied to a constant arbitrary number ($>1$). Otherwise, it will be divided with another constant value ($>1$). For practical purposes, if $\Lambda$ takes a value lower than a constant small arbitrary number, we set it equal to zero. If the $\Lambda$ is zero, and the chi-square is less than an arbitrary threshold value, we declare the \textbf{convergence} of the algorithm.\par
A flow chart summary of this method is shown in \ref{fig:LM_flow}. The Python code for LM is also available in appendix \ref{chap:appendix,sub:LM}.
\begin{figure}[h!]
\centering
\includegraphics[scale =0.9]{LM_flow.png}
\caption{Flow chart of Levenberg-Marquardt algorithm}
\label{fig:LM_flow}
\end{figure}
%--------------------------------------------------------------------------------
\section{Markov Chain Monte Carlo (MCMC)}
Given that Levenberg-Marquardt (LM) is not always effective in finding the best-fit point for complex likelihood spaces, a more powerful algorithm such as Markov Chain Monte Carlo (MCMC) is necessary in many real-life applications. \par
MCMC is particularly effective in fitting non-Gaussian likelihood spaces and has a guaranteed convergence, regardless of the starting point in parameter space. However, the algorithm is computationally heavy due to its iterative process, which is based on the evaluation of chi-square on each step. Initially, an arbitrary point in parameter space is given to MCMC as its initial trial step and the associated chi-square is calculated. Then, a random point is drawn from a Gaussian distribution, where the mean is set at the last point in the chain\footnote{In Python applications, the numpy.randn function is used to serve this purpose.}. Subsequently, the new chi-square is compared to the previous one from the last trial step. If the new chi-square is lower, the new trial point is accepted in the chain. If the new chi-square is higher, the trial point is accepted with a probability determined by a specific criterion.\par
The above-mentioned probability threshold is typically defined as:
\begin{equation}
    Probability = e^{\frac{-1}{2}(\chi_{new}^2 - \chi^2)}
\end{equation}
Which again has Gaussian insights.\par
As a measure of MCMC performance, the acceptance ratio is used to determine the fraction of trial steps that end up getting accepted into the chain. An ideal MCMC would typically have an acceptance ratio of 25 percent. However, even with a lower acceptance ratio, the MCMC can still converge, but it will require more trial steps.\par
A visual summary of MCMC algorithm is shown in \ref{fig:MCMC_flow}, and the Python code is available at \ref{chap:appendix,sub:MCMC}.
\begin{figure}[h!]
\centering
\includegraphics[scale =0.9]{MCMC_flow.png}
\caption{Flow chart of MCMC algorithm}
\label{fig:MCMC_flow}
\end{figure}
%-----------------------------------------------------------------------------
\subsection{Convergence Test}
The MCMC algorithm is designed to explore different regions of the parameter space in order to reach convergence. Various methods have been developed to ensure that the MCMC has converged, one of which is to check the power spectrum.\par
The power spectrum represents the distribution of power at different frequencies in the MCMC chain. A converged MCMC chain must have the behaviour of a white noise, with power uniformly distributed among all frequencies. On the other hand, an unconverged chain will show more power at lower frequencies compared to higher ones. Therefore, the criterion for checking the convergence of an MCMC chain is the flatness of the power spectrum in low frequencies when plotted on a log-log scale. Figure \ref{fig:MCMC_unconverged} and \ref{fig:MCMC_converged} illustrates the difference between a converged and an unconverged chain.\par
\begin{figure}[h!]
\centering
\includegraphics[scale =0.9]{mcmc_uncoverged.jpg}
\caption[An unconverged MCMC chain and its power spectrum]{An unconverged MCMC chain (left panel) and its power spectrum (right panel): The power tend to increase in lower frequencies, the chain itself does not indicate the behaviour of white noise, plot from Jonathan Sievers}
\label{fig:MCMC_unconverged}
\end{figure}

\begin{figure}[h!]
\centering
\includegraphics[scale =0.9]{mcmc_converged.jpg}
\caption[A converged MCMC chain and its power spectrum]{A converged MCMC chain (left panel) and its power spectrum (right panel): Note the flatness on low frequencies, plot from Jonathan Sievers}
\label{fig:MCMC_converged}
\end{figure}
%------------------------------------------------------------------------------------------------------
\section{Combination of MCMC and LM}
As mentioned before, MCMC is a computationally heavy algorithm due to its iterative nature. If calculating the chi-square takes a long time on each step, the MCMC itself will have a rather long run time before reaching the converged state. Different methods have been proposed to deal with this issue and help the MCMC to converge faster, one of which is to use the insights from running LM.\par
We previously discussed that parameters of a model might be correlated (\ref{chap:method,sub:cov_mat}). During a simple MCMC, we are drawing random samples from a gaussian distribution. This samples do not take the possible correlations into account. However, if we generate samples with such characteristics, the probability of them getting accepted into the actual chain is higher. Therefore, this approach (feeding the MCMC with a posterior distribution) will eventually assist the MCMC to explore more efficient regions of parameter space, and converge faster. Figure \ref{fig:combined_flow} illustrates this technique briefly.\par
Since the covariance matrix of these needed samples is already in hand, we are able to easily generate a set of correlated noise samples from that. This procedure is described in the following section.\par

\begin{figure}[h!]
\centering
\includegraphics[scale =0.9]{combined_flow.png}
\caption{Flow chart of the procedure to combine MCMC and LM}
\label{fig:combined_flow}
\end{figure}

\subsection{Generating correlated noise}
\label{chap:method,sub:correlated noise}
As discussed before, the off-diagonal elements of a covariance matrix correspond to the inverse of covariance between each pair of parameters. Thus, if we draw samples from the inverse of covariance matrix (which needs to be calculated at the point of "best-fit"), we are essentially sampling from the multivariate normal distribution with the deviation values describing the uncertainties in the parameters\footnote{Figure \ref{fig:corner_plots}, which shows the corner plots for a four-variable gaussian model, emphasizes the importance of sampling from a multivariate normal distribution and taking the correlations between the parameters into account.}.\par
The equivalent methods can be used to generate correlated noise: Cholesky and eigenvalue decomposition\footnote{Normally, calculating the Cholesky decomposition takes a shorter amount of time.}. For practical reasons, we prefer to use the eigenvalue decomposition for 21cm applications. The procedure is a s follows: A matrix of normal gaussian-drawn random variables is constructed in the desired shape ($n \times m$, corresponding to the number of samples and number of parameters respectively). Then, it is multiplied by the eigenvalue matrix and scaled by the square root of the eigenvalues. The transpose of the product will give us the correlated samples. \par
To gain more precision, is it possible to use the eigenvalues decomposition of the normalized covariance matrix (where diagonal samples all equal unity). The normalization process is done by multiplying the covariance matrix with its own diagonal. Eventually, the drawn samples need to be scaled by the square root of the diagonal matrix.\par
The Python implementation of the above-mentioned procedure is given in \ref{chap:appendix,sub:draw}.

\begin{figure}[h!]
\centering
\includegraphics[scale =0.6]{corner_plot_chain.png}
\caption[Corner plots of an MCMC chain]{Corner plots (parameter vs parameter) of a typical MCMC chain imposed on a four-variable gaussian model. The lower right panel clearly shows patterns of correlation between a pair of parameters. Sampling from a multivariate normal distribution based on the covariance matrix of these samples helps taking the correlations between these parameters into account.}
\label{fig:corner_plots}
\end{figure}

\section{Testing the Algorithm}
\label{chap:method,sub:test}
The complicated algorithm described in the previous sections, does not always behave as expected. Thus, naturally, one seeks options to weigh the output. In sections \ref{chap:method,sub:test,subsub:chi} and \ref{chap:method,sub:test,subsub:plot}, we introduce two methods to measure the overall quality of the model fitting.\par
\subsection{The chi-square test}
\label{chap:method,sub:test,subsub:chi}
This method is more focused on inspecting the output of LM and drawn samples. A large number of samples are drawn from the covariance matrix and the corresponding chi-squares are calculated. Since the covariance matrix describes the uncertainties in the parameters, we expect that the average difference in the chi-square statistics for two different samples should be of order unity per each parameter. This comes from the fact that the chi-square statistic scales with the uncertainties in the data and model, which are typically of order 1.\par
Figure \ref{fig:csq_test} shows an example of the distribution of difference between the chi-square values.\par
\begin{figure}[h!]
\centering
\includegraphics[scale =0.9]{chi-sqaure test.jpg}
\caption[Histogram of difference in the chi-square values of drawn samples]{Histogram illustrating the distribution of the values of difference between the chi-squares of drawn samples for a four variable model. The average is 3.97 and the standard deviation is 2.83, in agreement with our expectations. The best-fit parameters is considered as a "good fit".}
\label{fig:csq_test}
\end{figure}
\subsection{Chi-Square vs Parameters Plots}
\label{chap:method,sub:test,subsub:plot}
Another method to verify the results is to plot the chi-square values of drawn samples versus the each of the parameters. According to the definition of chi-square \ref{eq:chi-square matrix} (the non-linear dependency of chi-square on model parameters) we expect to observe a parabolic behaviour.\par
Figure \ref{fig:csq_params} demonstrates chi-square vs parameters plots for the same model as \ref{fig:csq_test}.
\begin{figure}[h!]
\centering
\includegraphics[scale =0.9]{csq_params.jpg}
\caption[Chi-Square vs parameters plots]{Plots of the values of chi-square of drawn samples versus the values of parameters for a four variable model. The parabolic behaviour is obvious.}
\label{fig:csq_params}
\end{figure}
%###################################################################################
\chapter{Results and Analysis}
\label{chap:results}
Reminder: citations for physical explanation of parameters\par
Chapter \ref{chap:method} carefully described a specific method to fit an experimental data set to its corresponding theoretical model. In this chapter, we are going to impose this method to actual global 21cm curves. We chose to use ARES as our simulator to generate these curves. \par
As discussed in chapter \ref{chap:global21cm}, the physical model of 21cm curve is dependent on a large number of parameters. However, this model can be effectively described using the following parameters:
\begin{enumerate}
    \item \textbf{$N_{lw}$}: Number of photons emitted in the Lyman-Werner band ($11.2-13.6eV$) per baryon of star formation (This parameter is referred to as \emph{pop\_rad\_yield\_0\_} in ARES documentation)
    \item \textbf{$N_{ion}$}: Mean number of ionizing photons produced per baryon of star formation (This parameter is referred to as \emph{pop\_rad\_yield\_2\_} in ARES documentation)
    \item \textbf{$f_{esc}$}: Fraction of ionizing photons that escape their host galaxy into the IGM
    \item \textbf{$f_X$}: High-redshift normalization factor in the relation between X-ray luminosity and SFR
\end{enumerate}
Figure \ref{fig:sensivity} shows the effect of changing these parameters on an actual 21cm curve.\par

We take the above-mentioned physical parameters as fitting parameters and we try to find their best-fit values and corresponding error-bars.\par
%-----------------------------------------------------------------------------------
\section{Parameter Estimation of an ARES generated curve}
\label{chap:results,sub:known_curve}
Before using our developed script to fit actual data, we use it to fit a known ARES curve as a verification test. We generate an ARES curve with a fixed set of parameters and we use these curve as the imaginary "data". If the algorithm returns the same parameter values (with in the error-bar range), we can make sure that it is working correctly.\par
We begin by using the LM and we find inverse of covariance matrix for our chosen combination of parameters, which we will later use to generate our correlated samples.\par
Parameters:
\begin{equation}
\emph{pop\_rad\_yield\_0\_}: 10^4, \quad \emph{pop\_rad\_yield\_2\_}: 10^3,\quad f_{esc}: 0.1,\quad f_X: 0.1 
\label{eq:parameter_values}
\end{equation}
Inverse of covariance matrix for this set of parameters:
\begin{equation}
\label{eq:cov_mat_known_curve}
    \begin{pmatrix}
    1.64\times 10^{-3} & 1.96 \times 10^{-2} &  -4.39 \times 10^{-7} &  -5.37 \times 10^{-8} \\
    1.96 \times 10^{-2} &  1.60 \times 10^{1} &  -1.44 \times 10^{-3} &  -5.94 \times 10^{-6} \\
    -4.39 \times 10^{-7} &  -1.44 \times 10^{-3} &  1.38 \times 10^{-7} &  2.15 \times 10^{-10} \\
    -5.37 \times 10^{-8} &  -5.94 \times 10^{-6} &  2.15 \times 10^{-10} &  1.38 \times 10^{-11} 
    \end{pmatrix}
\end{equation}
Now, we are prepared to draws our samples based on method described in \ref{chap:method,sub:correlated noise}. Figure \ref{fig:histogram_samples_known_curve} illustrates the distribution of samples for the above-mentioned four parameters. As a sanity check, we calculate the mean and standard deviation of these distribution. We expect the mean to be close to values of \ref{eq:parameter_values}. Table \ref{tab:samples_mean_known_curve} represents the values of mean and standard deviation, which is in agreement with the expectations.\par

\begin{table}
\centering
\caption[Mean and standard deviation of samples]{Mean and standard deviation of samples drawn from the covariance matrix \ref{eq:cov_mat_known_curve}}
\label{tab:samples_mean_known_curve}
\begin{tabular}{|c|c|c|c|c|}
\hline
\diagbox{Value}{Parameter} & \emph{pop\_rad\_yield\_0\_} & \emph{pop\_rad\_yield\_2\_} & \emph{$f_{esc}$} & \emph{$f_X$}\\
\hline
Mean & $9.99999993 \times 10^{3}$ & $1.00002933 \times 10^{3}$ & $9.99962873 \times 10^{2}$ & $1.00000029 \times 10^{1}$\\
\hline
Standard Deviation & $4.08174690 \times 10^{2}$ & $4.03138658 \times 10^{0}$ & $3.74407952 \times 10^{4}$ & $3.72581029 \times 10^{6}$\\
\hline
\end{tabular}
\end{table}


In section \ref{chap:method,sub:test}, we introduced two methods to check the algorithm and the quality of fit. Performing the chi-square test on our samples showed that the average difference between the chi-square of drawn samples is 49.20, which we expected to be ~4. Although it is approximately one order of magnitude bigger that the expectations, it can be still considered a good fit.\par

We proceed by performing the second test, which is plotting the values of ch-square of samples versus the parameter values. The results are shown in \ref{fig:csq_hist_known_curve}. In the hypothetical ideal case, We expected to see a parabolic behaviour. Figure \ref{fig:csq_vs_params_zoomed_known_curve} shows the zoomed version of results. We can infer a parabolic behaviour in low values of chi-square.\par

Now that we have our samples, we proceed to run the MCMC chain with them. The overall results of the fit and the final resulting curve are shown in \ref{tab:mcmc_results_known_curve} and \ref{} respectively.
Figure \ref{fig:chain_known_curve} indicates the white-noise behaviour of parameters throughout the chain which is consistent with our anticipation for a converged chain. We confirm the convergence by looking at the power spectrum in Figure \ref{fig:power_spectrum_known_curve}. Flat behaviour in low frequencies is apparent.\par


\begin{table}
\centering
\caption[Results of fitting a known ARES curve with MCMC chain]{Results of fitting a known ARES curve with MCMC chain}
\label{tab:mcmc_results_known_curve}
\begin{tabular}{|c|c|c|c|c|}
\hline
\diagbox{Value}{Parameter} & \emph{pop\_rad\_yield\_0\_} & \emph{pop\_rad\_yield\_2\_} & \emph{$f_{esc}$} & \emph{$f_X$}\\
\hline
True values & $1 \times 10^ {4}$ & $1 \times 10^ {3}$ & 0.1 & 0.1\\
\hline
Fitted values & $9.99998275 \times 10^ {3}$ & $9.99419824 \times 10^ {2}$ & $1.00025031 \times 10^ {-1}$ & $1.00001169 \times 10^ {-1}$ \\
\hline
Error-bar of fit & $3.54171104 \times 10^ {-2}$ & $3.86126208 \times 10^ {0}$& $3.57822431 \times 10^ {-4}$ & $3.39295495 \times 10^ {-6}$ \\
\hline
Fitting Error & 0.00013093\% & 0.02645403 \%& 0.01527014\%& 0.00032365\%\\
\hline
\end{tabular}
\end{table}
%------------------------------------------------------------
\section{Choice of "multiplication factor"}
\label{chap:results,sub:m_factor}
%-----------------------------------------------------------------------------------
\section{Parameter estimates of EDGES data and uncertainties}
\subsection{error bar calculation}
Reminder: not exactly sure where this equation comes from\par

We need to know the error bar of the observational data (EDGES) to run the MCMC chain. These error bars are usually taken from the noise model. In order to make our life easy, instead of dealing with the EDGES noise model which is hard and messy, we use the following straight forward equation which comes from the radiometer equation:\par
\begin{equation}
    \frac{\delta T}{T_{sys}} = \frac{1}{\sqrt{Bt}}
\end{equation}
Where B is the bandwidth, t is the exposure time and $T_{sys}$ is the system temperature. Assuming the bandwidth of $10^6$, exposure time of one day and and temperature of 3000K, we get:
$\delta T = 10 ^{-2}K = 10mK$
\subsection{results}
We proceed by doing the same procedure as section \ref{chap:results,sub:known_curve}. However, this time we use the EDGES data (with half of the actual amplitude) as our observational data in MCMC. This half amplitude is chosen using the insights from section \ref{chap:results,sub:m_factor}. One of the main issues is that since the shape of the EDGES data is not very close to the predicted theoretical curve, the resulting fit is not very perfect. We first run the LM to find the covariance matrix and generate our samples. 
\begin{equation}
\label{eq:cov_mat_edges}
    \begin{pmatrix}
    3.24\times 10^{4} & 5.68 \times 10^{6} &  -8.51 \times 10^{2} &  5.65 \times 10^{-1} \\
    5.68 \times 10^{6} &  2.43 \times 10^{10} &  -3.63 \times 10^{6} &  5.64 \times 10^{0} \\
    -8.51 \times 10^{2} &  -3.63 \times 10^{6} &  5.44 \times 10^{2} &  -9.80 \times 10^{-4} \\
    5.65 \times 10^{-1} &  5.64\times 10^{00} &  -9.80\times 10^{-4} &  3.92 \times 10^{-5} 
    \end{pmatrix}
\end{equation}

\begin{table}
\centering
\caption[Mean and standard deviation of samples]{Mean and standard deviation of samples drawn from the covariance matrix \ref{eq:cov_mat_edges}}
\label{tab:samples_edges}
\begin{tabular}{|c|c|c|c|c|}
\hline
\diagbox{Value}{Parameter} & \emph{pop\_rad\_yield\_0\_} & \emph{pop\_rad\_yield\_2\_} & \emph{$f_{esc}$} & \emph{$f_X$}\\
\hline
Mean & $4.54933238 \times 10^{3}$ & $2.47654876 \times 10^{3}$ & $3.70006283 \times 10^{-1}$ & $1.36397845 \times 10^{-1}$\\

\hline
Standard Deviation & $1.81616137 \times 10^{-1}$ & $1.56788160 \times 10^{2}$ & $2.34431702 \times 10^{-2}$ & $6.32705006 \times 10^{-6}$\\

\hline
\end{tabular}
\end{table}
We again run the MCMC from the drawn samples. The results are shown in table \ref{tab:mcmc_results_known_curve}. Figure \ref{fig:csq_trend_knwon_curve} and \ref{fig:corner_plots_known_curve} present the trend of values of chi-square throughout the chain and corner plots of parameters.Figure \ref{fig:csq_trend_knwon_curve} and \ref{fig:corner_plots_known_curve} present the trend of values of chi-square throughout the chain and corner plots of parameters.\par

\begin{table}
\centering
\caption[Results of fitting EDGES data with MCMC chain]{Results of fitting EDGES data with MCMC chain}
\label{tab:mcmc_results_known_curve}
\begin{tabular}{|c|c|c|c|c|}
\hline
\diagbox{Value}{Parameter} & \emph{pop\_rad\_yield\_0\_} & \emph{pop\_rad\_yield\_2\_} & \emph{$f_{esc}$} & \emph{$f_X$}\\
\hline
Fitted values & $9.99998275 \times 10^ {3}$ & $9.99419824 \times 10^ {2}$ & $1.00025031 \times 10^ {-1}$ & $1.00001169 \times 10^ {-1}$ \\
\hline
Error-bar of fit & $3.54171104 \times 10^ {-2}$ & $3.86126208 \times 10^ {0}$& $3.57822431 \times 10^ {-4}$ & $3.39295495 \times 10^ {-6}$ \\
\hline
\end{tabular}
\end{table}
%-----------------------------------------------------------------------------------
\section{Comparison with previous studies and observations}
I will write this section after the literature review
% -------------------------------------------------------------------------------------
\begin{figure}[h!]
\centering
\includegraphics[scale =0.5]{sensivity.png}
\caption[Behaviour of global 21cm model with respect to chosen parameters]{Behaviour of ARES-generated global 21cm models with respect to four parameters: \emph{pop\_rad\_yield\_0\_}, \emph{pop\_rad\_yield\_2\_}, $f_{esc}$, and $f_X$. In each panel, the corresponding parameter posses 10 different values and all other parameters are kept at default values by ARES}
\label{fig:sensivity}
\end{figure}

\begin{figure}[h!]
\centering
\includegraphics[scale =0.5]{histograms_known_curve.png}
\caption[Histogram of distribution of samples]{Histogram of distribution of samples before feeding to MCMC, The mean of these distributions are consistent with values given in \ref{eq:parameter_values}.}
\label{fig:histogram_samples_known_curve}
\end{figure}

\begin{figure}[h!]
\centering
\includegraphics[scale =0.7]{csq_hist_known_curve.png}
\caption[Histogram of Chi-Square of drawn samples]{Histogram of Chi-Square of drawn samples, the mean is 49.20 with standard deviation of 41.98.}
\label{fig:csq_hist_known_curve}
\end{figure}

\begin{figure}[h!]
\centering
\includegraphics[scale =0.5]{csq_vs_params_knwon_curve.png}
\caption[Chi-Square of Drawn samples vs parameter values]{Chi-Square of Drawn samples versus parameter values for a known ARES curve with four parameters, The parabolic behaviour is not obvious in these plots.}
\label{fig:csq_vs_params_knwon_curve}
\end{figure}

\begin{figure}[h!]
\centering
\includegraphics[scale =0.5]{csq_vs_params_zoomed_known_curve.png}
\caption[Chi-Square of Drawn samples vs parameter values, zoomed]{Zoomed version of figure \ref{fig:csq_vs_params_knwon_curve}, A weak parabolic behaviour is observed for low values of $\chi^2$.}
\label{fig:csq_vs_params_zoomed_known_curve}
\end{figure}

\begin{figure}[h!]
\centering
\includegraphics[scale =0.5]{chain_known_curve.png}
\caption[Trend of parameters]{Trend of parameters in the MCMC chain, White noise behaviour is considered as a sign of convergence. The acceptance ratio of the chain is $22.12\%.$}
\label{fig:chain_known_curve}
\end{figure}

\begin{figure}[h!]
\centering
\includegraphics[scale =0.5]{power_spectrum_known_curve.png}
\caption[Power spectrum of the chain]{Power spectrum of the chain, Flat behaviour in low frequencies is in agreement with results of \ref{fig:chain_known_curve}.}
\label{fig:power_spectrum_known_curve}
\end{figure}

\begin{figure}[h!]
\centering
\includegraphics[scale =0.6]{csq_trend_knwon_curve.png}
\caption[Trend of chi-square]{Trend of chi-square in the MCMC chain, The chain seems to have reached an equilibrium state.}
\label{fig:csq_trend_knwon_curve}
\end{figure}

\begin{figure}[h!]
\centering
\includegraphics[scale =0.6]{corner_plots_known_curve.png}
\caption[Corner plots of the chain]{Corner plots of the chain indicating the correlation between the parameters specially in the lower left panel}
\label{fig:corner_plots_known_curve}
\end{figure}
%@@@@@@@@@@@@@@@@@@@@@@@@@@@@@@@@@@@@@@@@@@@@@@@@@@@@@@@@@@@@@@@@@@@@@@@@@@
\begin{figure}[h!]
\centering
\includegraphics[scale =0.5]{histograms_edges.png}
\caption[Histogram of distribution of samples]{Histogram of distribution of samples before feeding to MCMC, The mean of these distributions are consistent with values given in \ref{eq:parameter_values}.}
\label{fig:histogram_samples_known_curve}
\end{figure}

\begin{figure}[h!]
\centering
\includegraphics[scale =0.7]{csq_hist_edges.png}
\caption[Histogram of Chi-Square of drawn samples]{Histogram of Chi-Square of drawn samples, the mean is 49.20 with standard deviation of 41.98.}
\label{fig:csq_hist_known_curve}
\end{figure}

\begin{figure}[h!]
\centering
\includegraphics[scale =0.5]{csq_vs_params_edges.png}
\caption[Chi-Square of Drawn samples vs parameter values]{Chi-Square of Drawn samples versus parameter values for a known ARES curve with four parameters, The parabolic behaviour is not obvious in these plots.}
\label{fig:csq_vs_params_knwon_curve}
\end{figure}

\begin{figure}[h!]
\centering
\includegraphics[scale =0.5]{chain_edges.png}
\caption[Trend of parameters]{Trend of parameters in the MCMC chain, White noise behaviour is considered as a sign of convergence. The acceptance ratio of the chain is $22.12\%.$}
\label{fig:chain_known_curve}
\end{figure}

\begin{figure}[h!]
\centering
\includegraphics[scale =0.5]{power_spectrum_edges.png}
\caption[Power spectrum of the chain]{Power spectrum of the chain, Flat behaviour in low frequencies is in agreement with results of \ref{fig:chain_known_curve}.}
\label{fig:power_spectrum_known_curve}
\end{figure}

\begin{figure}[h!]
\centering
\includegraphics[scale =0.6]{csq_trend_edges.png}
\caption[Trend of chi-square]{Trend of chi-square in the MCMC chain, The chain seems to have reached an equilibrium state.}
\label{fig:csq_trend_knwon_curve}
\end{figure}

\begin{figure}[h!]
\centering
\includegraphics[scale =0.5]{corner_plots_edges.png}
\caption[Corner plots of the chain]{Corner plots of the chain indicating the correlation between the parameters specially in the lower left panel}
\label{fig:corner_plots_known_curve}
\end{figure}
%###################################################################################
\chapter{Discussion and Conclusion}
\label{chap:discussion}
\section{Interpretation of the results}
%\section{Implications for cosmology and astrophysics}
\section{Summary of the main findings}
\section{Contributions and significance of the research}
\section{Limitations}
We faced certain limitation while working on this research. Most of those emerged during the implementation of python script. 
The method described in chapter \ref{chap:method}, is flexible to the change of parameters included in ARES's documentation\footnote{ARES documentation can be found \hyperlink{https://ares.readthedocs.io/en/latest/}{here}}. However, these parameters are only physically meaningful in a limited region of phase space. Going beyond this safe region will result in a failure in calculating the global 21cm curve due to ARES's inability to handle the combination of parameters. The code presented in this thesis is designed to handle this issue and stop it from aborting the whole run \footnote{The developed script for the MCMC is designed such that it will return an arbitrary large chi-square for those combination of parameters which ARES is unable to handle. Therefore, the probability of these points getting accepted into the chain will be systematically low.}, but the MCMC might not be able to reach convergence if the algorithm faces a lot of samples beyond the allowed region.\par
Another strongly limiting obstacle is the ARES's run time which is in the order of a few seconds per each simulation (approximately 4 to 5 seconds for different combination of parameters). Given than ARES is computationally heavy, it will make it hard to use algorithms based on using ARES on each iteration \footnote{ As an example, the run time for the MCMC results presented in chapter \ref{chap:results} was approximately 13 hours.}. This was main reason that led us to combine the MCMC with LM in order to have a more efficient chain and guarantee the convergence.\par 
%----------------------------------------------------------------------------
\section{Future work}
As the first step, the developed script is going to be published as a python package in near future. There will be a corresponding paper describing a summary of the results shown in chapter \ref{chap:results}.\par
We plan to take the analysis further and include a realistic foreground model. Then, we will add some of the previously mentioned non-standard effects to ARES simulations and run the same described method to observe the difference the best-fit curve. The results will hopefully tell us if EDGES observational data is a sign of new physics and can be explained through physics beyond standard cosmological model.\par

%##############################################################################
\chapter{Appendices}
\section{Derivation of radiometer equation}
\section{Code snippets and scripts}	
\subsection{Levenberg-Marquardt}
\label{chap:appendix,sub:LM}
\subsection{Markov Chain Monte Carlo}
\label{chap:appendix,sub:MCMC}
\subsection{drawing samples from covariance matrix}
\label{chap:appendix,sub:draw}
\begin{lstlisting}[language=Python, caption=Python example]
    def draw_samples(covariance_matrix, nset):
    """
    covariance_matrix: covariance matrix
    nset:the number of samples
    returns: a matrix of samples
    This function calculates a series of correlated samples based on the presented 
    covariance matrix and the number of samples.
    The shape of the output is (nset, m) where m comes from the shape of 
    covariance matrix and it typically shows the number of parameters in the model.
    """
    #normalizing the covariance matrix
    D = np.diag(np.diag(covariance_matrix)) #diagonal matrix of covariance matrix
    D_sqrt = np.sqrt(D)
    D_inv_sqrt = np.linalg.pinv(D_sqrt)
    #normalized covariance matrix
    covariance_matrix_normalized = D_inv_sqrt @ covariance_matrix @ D_inv_sqrt 

    e,v = np.linalg.eigh(covariance_matrix_normalized)
    e[e<0]=0 #omitting any negative eigenvalues due to roundoff
    n = len(e)

    #make gaussian random variables
    g=np.random.randn(n, nset)

    #scaling by the square root of the eigenvalues
    rte=np.sqrt(e)
    for i in range(nset):
        g[:,i]=g[:,i]*rte

    #calculating the samples
    samples = (v@g).T
    #denormalizing the samples
    samples_denormalized = samples @ D_sqrt
    return samples_denormalized
\end{lstlisting}
	% Begin Bibliography
	{
	
	\bibliography{references}
	\bibliographystyle{ieeetr}
	
	}
\end{document}